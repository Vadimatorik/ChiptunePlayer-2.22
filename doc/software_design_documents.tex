\documentclass[a4paper, 12pt]{article}
\usepackage[left=25mm, top=20mm, right=10mm, bottom=20mm, nohead, nofoot]{geometry}

\usepackage [warn]{mathtext}				% Чтобы можно было использовать русские буквы в формулах, 
											% но в случае использования предупреждать об этом
\usepackage{placeins}
\usepackage [T2A]{fontenc}		            % Выбор внутренней TEX−кодировки.
\usepackage [utf8]{inputenc}		        % Выбор кодовой страницы документа.
\usepackage [english, russian]{babel}		% Выбор языка документа.
\usepackage{amsmath}						% Математика.
\usepackage{svg}
\usepackage{graphicx}						% Картинки.
\usepackage{xcolor}
\usepackage{indentfirst}					% Красная строка в начале абзаца.
\usepackage{svg}

											%Настраиваем гиппер-ссылки.
\usepackage[pdfpagelayout=OneColumn, 		% pdf отображается как сплошная полоса из A4.
			colorlinks=true,				% Не нужно рисовать рамку вокруг ссылок, но при этом идет выделение цветом.
			linkcolor=black					% Используем черный цвет для обозначения гиппер ссылок в оглавлении.					
]{hyperref} 

\begin{document}

\title {ОПИСАНИЕ ПРОГРАММНОГО ОБЕСПЕЧЕНИЯ CHIPTUNE PLAYER 2.22}
\author {Автор: Дерябкин Вадим}
\date {2018}
\maketitle
\clearpage

\tableofcontents							% Оглавление.
\clearpage									% Первая глава должна идти начиная со следущей страницы.

\section{Принятые соглашения об именах в проекте}
\subsection{Именование файлов}
В проекте каждому файлу .cpp соответствует файл .h.

Файлы именуются по по шаблону:
\begin{verbatim}
ayplayer_name.cpp
ayplayer_name.h
\end{verbatim}

\textbf{name} - имя файла.\\

\textbf{Пример}: \textit{core} из описания эквивалентно \textit{ayplayer\_core.cpp} и \textit{ayplayer\_core.h} в древе проекта.

\subsection{Правила формирования имен глобальных объектов}
В коде программы именование глобальных структур-констант, передаваемых конструктору глобального объекта производится по шаблону:
\begin{verbatim}ayplayer_name_cfg\end{verbatim}
\textbf{name} - имя объекта, который будет инициализирован при помощи данной структуры.\\\\
\textbf{Пример}: \textit{lcd\_res\_pin} из описания эквивалентно \textit{ayplayer\_lcd\_res\_pin\_cfg} в коде.\\

В коде программы именование глобальных объектов производится по шаблону:
\begin{verbatim}ayplayer_name_obj\end{verbatim}
\textbf{name} - имя объекта.\\\\
\textbf{Пример}: \textit{lcd\_res\_pin} из описания эквивалентно \textit{ayplayer\_lcd\_res\_pin\_obj} в коде.


\section{Глобальные объекты}

\subsection{Объект глобального порта}
\noindentИмя объекта: \textbf{gp}\\
\noindentЭкземпляр класса: \textbf{global\_port}\\
\noindentРасположение: user\_code -> mk\_hardware -> port\\

Данный объект агрегирует внутри себя все используемые выводы микроконтроллера, предоставляя возможность инициализации всех перечисленных в структуре инициализации выводов одной командой.

Включает в себя следующие выводы:
\begin{itemize}
	\item adc\_bat
	\item adc\_right
	\item adc\_left
	\item button\_inc - соединенный с клавишей увеличения громкости.
	\item button\_dec - соединенный с клавишей уменьшения громкости.
	\item midi\_uart\_rx
	\item lcd\_clk
	\item lcd\_pwm
	\item lcd\_mosi
	\item lcd\_res - подключен к выводу сброса LCD. Инвертированный.
	\item lcd\_dc - подключен к выводу выбора интерпретации пакетов <<Команда/Данные>>, принимаемых LCD по SPI.
	\item lcd\_cs - подключен к выводу выбора LCD в качестве приемника на шине SPI1.
	\item sd1\_push
	\item sd1\_smd
	\item sd1\_d0
	\item sd1\_d1
	\item sd1\_d2
	\item sd1\_d3
	\item sd1\_clk
	\item otg\_fs\_vbus
	\item usb\_dm
	\item usb\_dp
	\item bdir - подключен к выводам выбора направления работы параллельной шины обоих музыкальных сопроцессоров.
	\item ay\_1
	\item ay\_2
	\item bc1 - подключен к выводам-защелкам приема данных из параллельной шины.
	\item ay\_clk
	\item spi\_audio\_clk
	\item spi\_audio\_tx
	\item shdn - подключен к выводу перевода каналов цифрового потенциометра в Z состояние.
	\item spi\_audio\_st\_reg - подключен к выводам-защелкам внутренних сдвиговых регистров цифровых потенциометров.
	\item dp\_cs\_res - подключен к выводу выбора цифровых потенциометров в качестве приемника на шине SPI3.
	\item button\_in - соединенный с общим контактом векторной клавиатуры.
	\item swd\_io
	\item swd\_clk
	\item pwr\_5\_v\_on - подключен к транзистору, открывающему подачу 5 вольт на плату.
	\item pwr\_on - подключен к транзистору, разрешающему подачу входного напряжения на плату.
	\item tp\_st
	\item tp\_ch
	\item chip\_1\_pwr\_on - подключен к транзистору, открывающему подачу питания на первый чип.
	\item chip\_2\_pwr\_on - подключен к транзистору, открывающему подачу питания на второй чип.
	\item sd2\_rx
	\item sd2\_tx
	\item sd2\_clk
	\item sd2\_cs - подключен к выводу выбора системной карты памяти в качестве приемника на шине SPI2.
	\item sd2\_push
	\item boot\_tx
	\item boot\_rx
\end{itemize}

\subsection{Порты ввода-вывода}

\end{document}}